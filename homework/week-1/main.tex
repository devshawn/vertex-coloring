\documentclass{article}
\usepackage[utf8]{inputenc}
\usepackage{amsmath}
\usepackage{amsthm}
\usepackage[left=1in,right=1in,top=0.9in,bottom=0.9in]{geometry}
\usepackage{graphicx}
\usepackage{float}
\setlength\parindent{0pt}

\newtheorem*{definition}{Definition}
\newtheorem*{theorem}{Theorem}


\begin{document}

\section*{Heuristics}
In this report, I will use two vertex coloring heuristics as defined in class:
\begin{itemize}
\item \textbf{Heuristic A}: Greedy algorithm
\item \textbf{Heuristic B}: Greedy algorithm plus degree sequencing
\end{itemize}

Heuristic B is also known as the Welsh-Powell algorithm, which is defined in \cite{welsh}.

\section*{Construct a graph}
Construct a simple graph, \(G = (V, E)\), such that:

\begin{align}
|V| &\geq 8 \\
\chi(G) &> \chi^{*}(G)
\end{align}

Let \(\chi\) refer to the coloring generated by \emph{heuristic B}. Let \(\chi^{*}\) be the optimal coloring.

\subsection*{Example 1}
I constructed a simple graph, \(G = (V, E)\), such that \(\Delta(G) = 3\) and \(\delta(G) = 1\). I've let \(|V| = 8\) for this example.

\begin{figure}[H]
\centering
\includegraphics[scale=0.6]{graph-1.png}
\caption{Uncolored original graph \(G\)}
\end{figure}

By applying \emph{heuristic B}, we get the following coloring. The numbers indicate the ordering of vertices before applying the heuristic. Any vertex of the same degree got assigned arbitrarily. This results in \(\chi(G) = 3\).

\begin{figure}[H]
\centering
\includegraphics[scale=0.6]{graph-2.png}
\caption{Coloring from applying heuristic B}
\end{figure}

\begin{definition}[Bipartite graph]
A bipartite graph is one whose vertex set can be partitioned into two subsets X and Y such that each edge has one end in X and one end in Y
\end{definition}

We can see that this is a bipartite graph, defined above by \cite{bondymurty}. Thus, \(G\) is \emph{2-colorable}. This means \(\chi^{*}(G) = 2\). \(G\) is an example graph that satisfies conditions \((1)\) and \((2)\). The optimal coloring is shown below.

\begin{figure}[H]
\centering
\includegraphics[scale=0.6]{graph-3.png}
\caption{Optimal coloring of graph \(G\)}
\end{figure}

\section*{Upper Bounds}
Examine the upper bounds of the greedy heuristic and the greedy plus degree sequencing heuristic. \newline

To do so, let's create another example that satisfies conditions \((1)\) and \((2)\) from above. After doing some research into bipartite graphs, I learned that \emph{crown graphs} are excellent at showing how bad greedy heuristics can be, as shown and defined in \cite{kordecki}.

\begin{definition}[Crown Graph]
A crown graph \(CR_n = (V, E)\) is an undirected graph with two sets of vertices where \(V = V_1 \cup V_2\) with an edge from \(v_i \in V_1\) to \(v_{j} \in V_2\) whenever \(i \neq j\). A crown graph can also be viewed as a complete bipartite graph where the edges of a perfect matching have been removed.
\end{definition}

\subsection*{Example 2}

I constructed a simple crown graph, \(H = (V, E)\). I've let \(|V| = 10\) for this example.

\begin{figure}[H]
\centering
\includegraphics[scale=0.38]{graph-4.png}
\caption{Uncolored original graph \(H\)}
\end{figure}

We can see that \(\Delta(G) = 4\). We can also see \(d(v) = 4\) for all \(v \in V\). Thus, in both heuristics, the greedy algorithm would pick an order arbitrarily. We can show using this crown graph the worst-case scenario of these heuristics.

\begin{figure}[H]
\centering
\includegraphics[scale=0.38]{graph-5.png}
\caption{Worst-case coloring of \(H\) using either greedy heuristic}
\end{figure}

In Figure 5, using either heuristic with this ordering, we get \(\chi(H) = 5\). This gives us \(\frac{|V|}{2}\) colors. This is the worst-case for a crown graph as shown in \cite{johnson}, but this graph also demonstrates the upper bounds for these heuristics. \newline

\subsection*{In General}

Let \(P = (V, E)\) be a simple, complete graph. \newline

We can see that \(\chi(H) = \Delta(H) + 1\). This is very easy to see in a \emph{complete} graph, where all vertices are connected to every other vertex. This means all vertices have degree \(|V| - 1\). Thus, every time we color a node, a new color is needed. And since we have \(\Delta(P) = |V| - 1\) and \(|V|\) vertices, we will need \(\Delta(P) + 1\) colors. This is stated in \emph{Brooks' Theorem}.

\begin{theorem}[Brooks' Theorem]
For any connected undirected graph \(G\) with maximum degree \(\Delta\), the chromatic number of \(G\) is at most \(\Delta\) unless \(G\) is a complete graph or an odd cycle, in which case the chromatic number is \(\Delta + 1\).
\end{theorem}

The proof of \emph{Brook's Theorem} can be found in \cite{lovasz}. Overall, heuristic A and heuristic B can produce some very undesirable results. In graph \(H\), at the worst case, these heuristics produce \(\chi(H) = 5\) when \(\chi^{*}(H) = 2\) as it is bipartite. This is shown below.

\begin{figure}[H]
\centering
\includegraphics[scale=0.38]{graph-6.png}
\caption{Optimal coloring of \(H\)}
\end{figure}

\begin{thebibliography}{9}
\bibitem{bondymurty}
Bondy, J.A. and U.S.R. Murty [1976],
\emph{Graph Theory with Applications},
American Elsevier Publishing, New York, NY.

\bibitem{johnson}
Johnson, D. S. [1974], \emph{Worst-case behavior of graph coloring algorithms}, Proc. 5th Southeastern Conf. on Combinatorics, Graph Theory, and Computing, Utilitas Mathematicae, Winnipeg, pp. 513–527.

\bibitem{kordecki}
Kordecki, W. and A. Łyczkowska-Hanćkowiak [2016], \emph{Greedy online colouring with buffering}, arXiv preprint arXiv:1601.00252.

\bibitem{lovasz}
Lovász, L. [1975], \emph{Three short proofs in graph theory}, Journal of Combinatorial Theory, Series B, 19(3), 269-271.

\bibitem{welsh}
Welsh, D. J. and Powell, M. B. [1967], \emph{An upper bound for the chromatic number of a graph and its application to timetabling problems}, The Computer Journal, 10(1), 85-86.

\end{thebibliography}


\end{document}
